\chapter{Introduzione}

\section{Il progetto}

Il progetto da sviluppare è una web application in ambito Blockchain e NFT. In particolare il sito sviluppato si può definire come simile ad un e-commerce in quanto lo scopo finale è effettuare azioni di compravendita tra i vari utenti iscritti ad esso.\\
All'interno della web application l'utente può navigare nella pagina principale e visualizzare le opere multimediali poste in vendita, eseguire l'accesso se possiede un account nel sito oppure registrarsi come nuovo account.\\
Dopo essere acceduto al sito, l'utente può modificare i propri dati, inserire una nuova opera multimediale o modificare i dati di quelle precedentemente inserite e comprare un'opera multimediale caricata nel sito da un altro utente.\\
Essendo un progetto legato al concetto di blockchain e NFT le azioni di compravendita verranno effettuate tramite lo scambio di una moneta virtuale detta Ethereum. Un utente viene identificato tramite l'indirizzo del wallet con cui acquisterà o caricherà una nuova opera multimendiale.\\
Quando un utente carica un'opera multimediale, il file viene salvato tramite codice hash nella blockchain ed in questo modo le opere caricate saranno univoche. Infatti salvando queste informazioni nella blockchain viene creato un timestamp che funge da certificato di attribuzione dell'opera all'utente che carica o compra l'opera multimediale. In questo modo nessuno nessun altro utente può registrare la stessa opera a suo nome poichè una funzione hash produce sempre e solo un risultato rendendo uniche le opere multimediali.\\